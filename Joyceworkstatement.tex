\documentclass[12pt,letterpaper]{article}

\usepackage{amsmath, amsthm, amssymb, amsfonts}
\usepackage{graphicx}
\usepackage{bm}
\usepackage{natbib}

\theoremstyle{definition}
\newtheorem{dfn}{Definition}

\begin{document}

% The numbers below controls the amount of space between the following sections
\def\shiftdowna{0.32in}  % Adjust for balance
\def\shiftdownb{0.22in}  % Adjust for balance

% Set up the boiler plate at the top of the page

\begin{center}
\textbf{{\large Project Work Statement}}\\


% SPONSOR
\vspace \shiftdowna
\underline {Sponsor}\\ 
\vspace{5pt}
\textbf{{\large McDonald's Corporation}}\\


% TITLE
\vspace \shiftdowna
\textbf{{\large How much Ice do You need?}}


% STUDENTS
\vspace{0.35in}
\vspace \shiftdownb
\underline {Participants} \\
\vspace{5pt}
\text{Nam Lee}, \texttt{nhlee@jhu.edu}

% SPONSORS
\vspace \shiftdownb
\underline {Potential Participants}\\
\vspace{5pt}
Joyce Tan, \texttt{jtan21@jhu.edu} \\
\vspace{3pt}


% DATE
\vspace \shiftdowna
Date: \today

\end{center}

\vfill  
%Fill page to force following note to bottom
\footnoterule
\noindent \small{Any apparent association of this work to McDonalds is
fictional one, and the sole purpose of this work is a class exercise}

\newpage

\section{Background} 
McDonald's Corporation is the world's largest chain of hamburger fastfood restaurants, serving around 68 million customers daily in 119 countries. Mcdonald's primarily sells hamburgers, cheeseburgers, chicken, french fries, breakfast items, soft drinks, milkshakes and desserts. In response to healthier consumer taste, the company has expanded its menu to include salads, wraps,
smoothies and fruits.

\section{Problem Statement}

Selling soft drinks is a significant portion of McDonald's business, be it as a thirst quencher, or as part of the extra value meal. The server is not accustomed to putting much thought in measuring the amount of ice put in the cup. This often results in a overly diluted, overly concentrated or overly cold drink for the customer. This is likely to lower overall customer satisfaction, since a drink is a significant complement to a meal. Thus, customers are likely to appreciate if the right amount of ice was added for optimal satisfaction.

\section{Approach}
We are interested in approaching this problem from 2 different methods. The first method would be experimenting with different types of soda, and different amounts of ice to find out the optimal proportion of ice to soda. If time permits, we would be looking to approach from an alternative method. The second method would be using specific heat capacities of soda and ice, in order to calculate the optimal proportion for the iced soda to have optimal concentration as well as optimal temperature.

\section{Milestones}
We have the following major deadlines:
\begin{itemize}
    \item Work Statement due date, Sep 28, 2012,
    \item Midterm Presentation due date, Oct 12, 2012,
    \item Progress Report due date, Oct 26, 2012,
    \item Final Presentation due date, Nov 6, 2012,
    \item Final Report due date, Nov 30, 2012.
\end{itemize}

\section{Deliverable}
\subsection{From Team to Sponsor} % (fold)
The following outputs are expected from this project:
\begin{itemize}
    \item A table of optimal ratios for each different type of soda
    \item Matlab code with complete set of documentations that outputs optimal ratio based on specific heat capacities
    \item Numerical experiment results reporting success rate of different ice ratios
    \item Technical report and presentations summarizing the work. 
\end{itemize}

\subsection{From Sponsor to Team} % (fold)

In order for our project to be of successful one, we will need:
\begin{itemize}
    \item Types of soda used at McDonald's chains
    \item Computing resources
    \item Timely responses to inquiries, 
\end{itemize}


%\newpage
%\bibliographystyle{plain}
%%\renewcommand\bibname{Selected Bibliography Including Cited Works}
%\nocite{*}
%\bibliography{biblio}

\end{document}
